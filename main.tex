% !TEX TS-program = pdflatex
% !TEX encoding = UTF-8 Unicode

% This is a simple template for a LaTeX document using the "article" class.
% See "book", "report", "letter" for other types of document.

\documentclass[11pt]{article} % use larger type; default would be 10pt

\usepackage[utf8]{inputenc} % set input encoding (not needed with XeLaTeX)

%%% Examples of Article customizations
% These packages are optional, depending whether you want the features they provide.
% See the LaTeX Companion or other references for full information.

%%% PAGE DIMENSIONS
\usepackage{geometry} % to change the page dimensions
\geometry{a4paper} % or letterpaper (US) or a5paper or....
% \geometry{margin=2in} % for example, change the margins to 2 inches all round
% \geometry{landscape} % set up the page for landscape
%   read geometry.pdf for detailed page layout information

\usepackage{graphicx} % support the \includegraphics command and options

% \usepackage[parfill]{parskip} % Activate to begin paragraphs with an empty line rather than an indent

%%% PACKAGES
\usepackage{booktabs} % for much better looking tables
\usepackage{array} % for better arrays (eg matrices) in maths
\usepackage{paralist} % very flexible & customisable lists (eg. enumerate/itemize, etc.)
\usepackage{verbatim} % adds environment for commenting out blocks of text & for better verbatim
\usepackage{subfig} % make it possible to include more than one captioned figure/table in a single float
% These packages are all incorporated in the memoir class to one degree or another...

%%% HEADERS & FOOTERS
\usepackage{fancyhdr} % This should be set AFTER setting up the page geometry
\pagestyle{fancy} % options: empty , plain , fancy
\renewcommand{\headrulewidth}{0pt} % customise the layout...
\lhead{}\chead{}\rhead{}
\lfoot{}\cfoot{\thepage}\rfoot{}

%%% SECTION TITLE APPEARANCE
\usepackage{sectsty}
\allsectionsfont{\sffamily\mdseries\upshape} % (See the fntguide.pdf for font help)
% (This matches ConTeXt defaults)

%%% ToC (table of contents) APPEARANCE
\usepackage[nottoc,notlof,notlot]{tocbibind} % Put the bibliography in the ToC
\usepackage[titles,subfigure]{tocloft} % Alter the style of the Table of Contents
\renewcommand{\cftsecfont}{\rmfamily\mdseries\upshape}
\renewcommand{\cftsecpagefont}{\rmfamily\mdseries\upshape} % No bold!

%%% END Article customizations

%%% The "real" document content comes below...

\title{fiTQun RJMCMC}
\author{Richard Calland}
%\date{} % Activate to display a given date or no date (if empty),
         % otherwise the current date is printed 

\begin{document}
\maketitle

\section{Motivation}
\subsection{Current fiTQun method}
fiTQun in its current state constructs a joint likelihood from charge and time information, and maximizes this using the SIMPLEX routine~\cite{SIMPLEX}. In the case when there are multiple rings inside an event, the current strategy is to perform a series of fits increasing the number of rings each time, and use the resulting likelihood to decide which fit performed better. This method runs into problems as the number of rings increases, and in order to feasibly run this fit, the number of parameters are reduced by assuming all rings share a common vertex.

\subsection{Proposed changes}
Another approach to solving the multi-ring problem is to replace SIMPLEX with another method. By constructing the problem as a Mixture model, where each ring is a component of the mixture, we can sample the Posterior probability using a suitable method (that allows the addition/removal of components). The maximum of the Posterior probability will provide the most likely configuration of rings given a data set.

\section{The Mixture Model and Sampler}
A mixture model is a summation of PDFs that are weighted, such that the total weight always sums to 1: 

\begin{equation}
f_{total} = \sum_{i=1}^{i=k}\omega_{i}f(\theta_{i}) \qquad \sum_{i=1}^{i=k}\omega_{i} = 1
\end{equation}

where $\omega_{i}$ is the weight of component $i$, $\theta_{i}$ are the parameters describing component $i$, for a total of $k$ components. The constraint that all weights sum to 1 means that probability is conserved even when the number of components is changed.

\subsection{Reversible Jump Sampler}
In order to sample the mixture model, a trans-dimensional sampler must be used. This means that the sampler must be able to add or remove components from the mixture, effectively changing the number of parameters. 
The usual Markov chain Monte Carlo (MCMC) approach to sampling is to use the Metropolis-Hastings acceptance probability, accepting a new step with probability:

\begin{equation}
P_{accept} = min(1,a)
\end{equation}
where $a$ is defined as

\begin{equation}
a_{mh} = \frac{l(d|\theta')p(\theta')q(\theta|\theta')}{ l(d|\theta)p(\theta)q(\theta'|\theta)}
\end{equation}
where $\theta$ are the parameters of the current model, $\theta'$ the parameters of the proposed model, $l(d|...)$ the likelihood of data $d$, $p(...)$ the prior and $q(a|b)$ the proposal probability of moving to state $a$ from state $b$.

This holds for any samples that are proposed within the model. The Reversible Jump MCMC sampler allows jumps between models with different numbers of parameters, accepting said moves using:

\begin{equation}
a_{rj} = \frac{l(d|\theta')p(\theta')}{l(d|\theta)p(\theta)q(u)}|J|
\end{equation}
where there are dimension matching terms in the form of $q(u)$ to account for the different number of parameters of each model. In addition, a Jacobian term $J$ is used to adjust the probability measure according to the parameter transformations, however this term is often 1 and can be safely ignored.

To elaborate further on the reversible jump move, we will consider an example. Typically to add a new ring to the model, the new ring parameters are constructed using random numbers $u\sim g()$, which are then turned into the ring parameters via a transformation of the current model parameters $\theta_{new} = h(u,\theta)$.

\bibliography{bib}{}
\bibliographystyle{plain}
\end{document}
